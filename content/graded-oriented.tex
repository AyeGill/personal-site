---
title: Graded rings and complex orientations
date: 2018-12-07
---
A note about graded rings and complex orientable cohomology theories.

In the theory of complex oriented cohomology theories, one encounters a calculation of the following form:
We have a multiplicative cohomology theory $E$, with a given element $t \in E^2(\C P^\infty)$ which goes to the unit
in $E^2(\C P^1) = E^2(S^2)$ (a complex orientation).

One then shows that $E^*(\C P^n) \simeq E^*(\ast)[t]/(t^{n+1})$, with $t$ in degree $2$, and we compute
\[E^*(\C P^\infty) = E^*(\colim \C P^n) = \lim E^*(\C P^n) = \lim E^*(\ast)[t]/(t^{n+1}) = E^*(\ast)[[t]]\]
However, one can easily see that this should apply to
$E^*(-) = H^*(-;\Z)$, but also
\[H^*(\C P^\infty;\Z) = \Z[t] = H^*(\ast;\Z)[t]\]
So $\Z[[t]] = \Z[t]$.. but that's clearly wrong, right?

This apparent paradox is resolved by thinking more clearly about where the functor $E^*(-)$ takes values - the category of \emph{graded rings}.

\begin{definition}
  A \emph{$\Z$-graded ring} $R$ consists of this data:
  \begin{enumerate}
    \item A collection of abelian groups $\{R_n\}_{n\in\Z}$
    \item For each $n,m\in\Z$, a multiplication $\mu: R_n \times R_m \to R_{n+m}$
    \item Which is associative, i.e the two natural maps $R_n \times R_m \times R_k \to R_{n+m+k}$ agree for every $n,m,k\in \Z$.
    \item An element $1 \in R_0$
    \item So that $\mu(1,x) = \mu(x,1) = x$ for any $x\in R_n$, for any $n$.
  \end{enumerate}

  A \emph{homomorphism of $\Z$-graded rings} $R \to S$ consists of group homomorphisms $R_n \to S_n$, preserving the unit and the multiplication in the obvious sense.
\end{definition}

One can of course define a notion of commutativity (and \emph{graded commutativity}) in this setting as well, but this is not necessary for this discussion.
One will also, in general, need a theory of graded modules and algebras, but this is, again, unnecessary here.

\begin{example}
  If $E^*$ is a multiplicative cohomology theory, $R_n = E^n(X)$ defines a graded ring for any space $X$.
  Moreover, $E^*(Y) \to E^*(X)$ is a graded ring homomorphism for any map $X \to Y$ of spaces.
\end{example}

We have not defined polynomial or power series rings as graded rings yet, but the reason why $\Z[x] = \Z[[x]]$ might already be comprehensible:
they have the \emph{same} homogenous elements (a homogenous element of degree $k$ is just $n x^k$ for some $n\in\Z$).
In retrospect, it can even appear obvious that $\Z[x]$ can't mean something different than $\Z[[x]]$ when you're talking about a cohomology ring - after all, the cohomology theory only tells you what the homogenous elements are, the set of terms like $x^2 + 5x + 2$ is completely post hoc.

Let's give some definitions:
\begin{definition}
  Let $R$ be a $\Z$-graded ring.
  The polynomial ring $R[t]$ has the degree n piece $(R[t])_n$ consist of finite formal sums like $\sum_{k\geq 0} t^k x_k$, where $x_k \in R_{n-k}$.
  The product and sum are defined in the obvious way (exercise: check that the degrees work out),
  and the unit is $t^0 1 \in R[t]_0$


  The power series ring $R[[t]]$ has the degree n piece $(R[[t]])_n$ consist of (possibly infinite) formal sums like $\sum_{k\geq 0} t^k x_k$, where $x_k \in R_{n-k}$.
  The product and sum are defined in the obvious way (exercise: check that the degrees work out),
  and the unit is $t^0 1 \in R[[t]]_0$
\end{definition}

This clarifies two points:
$\Z[[x]] = \Z[x]$ because, since $\Z$ has nothing in negative degrees, the sums $\sum_{k\geq 0} t^k x_k$ must terminate when $k$ becomes greater than $n$ (and in fact the only nonzero term is when $k = n$).
In general, power series and polynomial rings in this setting agree when the degrees of the coefficient ring is bounded below.
This also answers a very reasonable followup question, namely: why write power series ring rather than polynomial ring?
The answer is that, in the case where $E^*(\ast)$ does have elements of unbounded negative degree, $E^*(\ast)[[t]]$ and $E^*(\ast)[t]$
differ (and it's $E^*(\ast)[[t]]$ that gives the correct answer).

An example of this is complex $K$-theory, where $K^0(\ast) = \Z = K^{2n}(\ast)$ (by Bott periodicity), and $K^1(\ast) = K^{2n+1}(\ast) = 0$ (exercise).
This implies that $K^0(\C P) \simeq \Z[[t]]$ (note that this is not a graded ring!).
